\chapter*{Project Summary}

Smartphone hardware has been developing at an ever increasing rate, particularly since 2007 when the Apple iPhone\textsuperscript{\texttrademark} was released. Since then to own a device with a CPU having a clockspeed in excess of one gigahertz and more than five-hundred and twelve megabytes of RAM is not uncommon. This project will attempt to explore the ability to perform complex image processing on such a device. 

The concept is to write a software program for these mobile environments that will attempt to identify which class of tree a photographed leaf originates. A user will startup the application on their smartphone and point it at the leaf, on clicking a button the phone will expose that image and begin attempting to identify the class of tree. When the software successfully identifies the tree the software will redirect the user to some knowledge source relating to that family of tree, for example Wikipedia.

A project of this type can be divided into two major components, the image processing mechanism and the mobile application. The image processing component will require a considerable portion of the time, and developing a successful solution will be difficult.

The second component is the mobile application into which the image processing will be integrated, while less challenging it will none the less consume a significant portion of time. Writing the application will require understanding of a number of components of the platform, notably interacting with the UI, accessing hardware devices and calling native code through the Native Development Toolkit. It is also important that the application fit the user expectations for this type of program in terms of functionality and usability.
